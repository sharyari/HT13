\documentclass[a4paper,10pt]{report}
\usepackage{fullpage}
\usepackage[british]{babel}
\usepackage[T1]{fontenc}
\usepackage{amsmath}
\usepackage{amssymb}
\usepackage[T1]{fontenc}
\usepackage[latin1]{inputenc} 
%\usepackage{amsthm} \newtheorem{theorem}{Theorem}
\usepackage{color}
\usepackage{float}



\usepackage{caption}
\DeclareCaptionFont{white}{\color{white}}
\DeclareCaptionFormat{listing}{\colorbox{gray}{\parbox{\textwidth}{#1#2#3}}}
\captionsetup[lstlisting]{format=listing,labelfont=white,textfont=white}


\usepackage{alltt}
\usepackage{listings}
 \usepackage{aeguill} 
\usepackage{dsfont}
%\usepackage{algorithm}
%\usepackage{algorithmicx}
\usepackage{subfig}
\lstset{% parameters for all code listings
	language=Python,
	frame=single,
	basicstyle=\small,  % nothing smaller than \footnotesize, please
	tabsize=2,
	numbers=left,
%	framexleftmargin=2em,  % extend frame to include line numbers
	%xrightmargin=2em,  % extra space to fit 79 characters
	breaklines=true,
	breakatwhitespace=true,
	prebreak={/},
	captionpos=b,
	columns=fullflexible,
	escapeinside={\#*}{\^^M}
}


% Alter some LaTeX defaults for better treatment of figures:
    % See p.105 of "TeX Unbound" for suggested values.
    % See pp. 199-200 of Lamport's "LaTeX" book for details.
    %   General parameters, for ALL pages:
    \renewcommand{\topfraction}{0.9}	% max fraction of floats at top
    \renewcommand{\bottomfraction}{0.8}	% max fraction of floats at bottom
    %   Parameters for TEXT pages (not float pages):
    \setcounter{topnumber}{2}
    \setcounter{bottomnumber}{2}
    \setcounter{totalnumber}{4}     % 2 may work better
    \setcounter{dbltopnumber}{2}    % for 2-column pages
    \renewcommand{\dbltopfraction}{0.9}	% fit big float above 2-col. text
    \renewcommand{\textfraction}{0.07}	% allow minimal text w. figs
    %   Parameters for FLOAT pages (not text pages):
    \renewcommand{\floatpagefraction}{0.7}	% require fuller float pages
	% N.B.: floatpagefraction MUST be less than topfraction !!
    \renewcommand{\dblfloatpagefraction}{0.7}	% require fuller float pages

	% remember to use [htp] or [htpb] for placement


\usepackage{fancyvrb}
%\DefineVerbatimEnvironment{code}{Verbatim}{fontsize=\small}
%\DefineVerbatimEnvironment{example}{Verbatim}{fontsize=\small}

\usepackage{url}
\urldef{\mailsa}\path|sharyari@gmail.com |    
\newcommand{\keywords}[1]{\par\addvspace\baselineskip
\noindent\keywordname\enspace\ignorespaces#1}


\usepackage{tikz} \usetikzlibrary{trees}
\usepackage{hyperref}  % should always be the last package

% useful colours (use sparingly!):
\newcommand{\blue}[1]{{\color{blue}#1}}
\newcommand{\green}[1]{{\color{green}#1}}
\newcommand{\red}[1]{{\color{red}#1}}

% useful wrappers for algorithmic/Python notation:
\newcommand{\length}[1]{\text{len}(#1)}
\newcommand{\twodots}{\mathinner{\ldotp\ldotp}}  % taken from clrscode3e.sty
\newcommand{\Oh}[1]{\mathcal{O}\left(#1\right)}

% useful (wrappers for) math symbols:
\newcommand{\Cardinality}[1]{\left\lvert#1\right\rvert}
%\newcommand{\Cardinality}[1]{\##1}
\newcommand{\Ceiling}[1]{\left\lceil#1\right\rceil}
\newcommand{\Floor}[1]{\left\lfloor#1\right\rfloor}
\newcommand{\Iff}{\Leftrightarrow}
\newcommand{\Implies}{\Rightarrow}
\newcommand{\Intersect}{\cap}
\newcommand{\Sequence}[1]{\left[#1\right]}
\newcommand{\Set}[1]{\left\{#1\right\}}
\newcommand{\SetComp}[2]{\Set{#1\SuchThat#2}}
\newcommand{\SuchThat}{\mid}
\newcommand{\Tuple}[1]{\langle#1\rangle}
\newcommand{\Union}{\cup}
\usetikzlibrary{positioning,shapes,shadows,arrows}

\pagestyle{empty}

\title{\textbf{Programming in RTOS using OSEK\\ and Lego Mindstorms NXT}}

\author{Jonathan Sharyari \and Daniel Tibbing \and Bj{\"o}rn Forsberg}

\begin{document}
\maketitle

\section*{Part 2: Event Driven Scheduling}

This part of the lab consists of writing an event driven scheduler that used a light sensor to stay on top of a table. To achieve this goal the robot initially  calibrates the light sensor against the color of the table. Note that this requires that the robot is on the table at system startup. 

The robot then polls the light value at regular intervals (every 100ms) and also displays this value on the LCD screen. If a light value differs from the calibrated value with more than 5 percent, the robot will stop regardless if the button is pressed or not. It will start again first when it is back on the table, and the button is released and pressed again.

The calibration is done by reading ten light values with 100ms intervals, and using the mean value as the calibrated value.

The software consists of two tasks, \texttt{EventdispatcherTask} and \texttt{MotorcontrolTask}. The \texttt{EventdispatcherTask} regularly updates the display with light values. Further, it polls the button to see if it is pressed or not. When the button is pressed, a \texttt{TouchOnEvent} is sent, and when the button is released a \texttt{TouchOffEvent}. It also sends events when the light conditions indicate that we are driving off the table. When the light sensor value indicates that we are driving off the table, the \texttt{LightOffEvent} is sent. Likewise, if the light sensor value changes to indicate that we are back on the table, the \texttt{LightOnEvent} is sent.

The \texttt{MotorcontrolTask} waits for the four events. When we recieve events we update the state of the \texttt{MotorcontrolTask} to reflect the recieved changes, and if the current state allows us to drive (button pressed and on table) we do that, otherwise we stop.


\appendix
\subsection*{Part 2: C-code}

\begin{lstlisting}[label=some-code,caption=part2.c,mathescape]
#include <stdlib.h>
#include <stdio.h>
#include "kernel.h"
#include "kernel_id.h"
#include "ecrobot_interface.h"

DeclareTask(MotorcontrolTask);
DeclareTask(EventdispatcherTask);
DeclareEvent(TouchOnEvent);
DeclareEvent(TouchOffEvent);
DeclareEvent(LightOnEvent);
DeclareEvent(LightOffEvent);

#define LIGHTSENSOR NXT_PORT_S1
#define TOUCHSENSOR NXT_PORT_S2
#define LIGHTVAL ecrobot_get_light_sensor(LIGHTSENSOR)
#define TOUCHVAL ecrobot_get_touch_sensor(TOUCHSENSOR)

void user_1ms_isr_type2(void) {}

void ecrobot_device_initialize() {
  ecrobot_set_light_sensor_active(LIGHTSENSOR);
}
void ecrobot_device_terminate() {
  ecrobot_set_light_sensor_inactive(LIGHTSENSOR);
}

/* This function will print the light intensity on the LCD */
void display_light_intensity() {
  display_clear(1);
  display_goto_xy(0,0);
  display_string("Light: ");
  display_unsigned(LIGHTVAL, 3);
  display_update();
}

/*  This function will take 10 light intensity values, waiting 100ms between each measurement
    and return the mean measurement value. This is used for initial calibration */
int light_calibration_proc() {
  int threshold = 0;
  for (int i = 0; i < 10; i++){
    threshold += LIGHTVAL;
    systick_wait_ms(10);
  }
  return threshold /= 10;
}

TASK(EventdispatcherTask){
  /* buttonState is TRUE when the button is pressed, else FALSE */ 
  int buttonState = FALSE;
  /* tableState is TRUE when we are on the table, otherwise FALSE */
  int tableState = TRUE;

  /* Threshold holds the initial light settings. The robot must be on the working surface when initialized */	
  int threshold = light_calibration_proc();
  
  while (TRUE) { /* Forever! */
    if (buttonState != TOUCHVAL) { /* If button state has changed */
      buttonState = !buttonState; /* State changed, so we update the variable */
      if (buttonState) /* If button pressed */
	SetEvent(MotorcontrolTask, TouchOnEvent); /* Signal motors ok to run */
      else
	SetEvent(MotorcontrolTask, TouchOffEvent); /* Signal motors must stop */
    }
    
    if (tableState != (LIGHTVAL >= threshold*1.05)){ /* If the table state has changed */
      tableState = !tableState;
      if(tableState)
	SetEvent(MotorcontrolTask, LightOffEvent); /* Signal motors must stop */
      else
	SetEvent(MotorcontrolTask, LightOnEvent); /* Signal motors ok to run */
    }
    display_light_intensity(); /* Update the light values on the display*/
    systick_wait_ms(100); /* Do we need this? */
  }
}

TASK(MotorcontrolTask) {
  
  int table = FALSE;
  int button = FALSE;
  int okToDrive = FALSE;
  
  EventMaskType eventmask = 0;
  
  while(true) {

    /* The only time we have to reevaluate if it is ok to run is when something changes, so wait for the events. */
    WaitEvent(TouchOnEvent | LightOnEvent | TouchOffEvent | LightOffEvent);
    GetEvent(MotorcontrolTask, &eventmask);

    /* Set the state of button and table according to the recieved event. */
    if(eventmask & TouchOnEvent) {
      button = TRUE;
      ClearEvent(TouchOnEvent);
    }
    else if(eventmask & TouchOffEvent) {
      button = FALSE;
      ClearEvent(TouchOffEvent);
    }
    else if(eventmask & LightOnEvent) {
      table = TRUE;
      ClearEvent(LightOnEvent);
    }
    else if(eventmask & LightOffEvent) {
      table = FALSE;
      ClearEvent(LightOffEvent);
    }

    /* Update the OK to drive flag if all prerequisites are met. */
    if(button && table && !okToDrive) {
      okToDrive = TRUE;
      nxt_motor_set_speed(NXT_PORT_A, 100, 0);
      nxt_motor_set_speed(NXT_PORT_B, 100, 0);
    } else if((!button || !table) && okToDrive) {
      okToDrive = FALSE;
      nxt_motor_set_speed(NXT_PORT_A, 0, 1);
      nxt_motor_set_speed(NXT_PORT_B, 0, 1);
    }

  }
}
\end{lstlisting}
\newpage
\subsection*{Part 2: Oil-file}

\begin{lstlisting}[label=some-code,caption=part2.oil,mathescape]
#include "implementation.oil"

CPU ATMEL_AT91SAM7S256
{
  OS LEJOS_OSEK
  {
    STATUS = EXTENDED;
    STARTUPHOOK = FALSE;
    ERRORHOOK = FALSE;
    SHUTDOWNHOOK = FALSE;
    PRETASKHOOK = FALSE;
    POSTTASKHOOK = FALSE;
    USEGETSERVICEID = FALSE;
    USEPARAMETERACCESS = FALSE;
    USERESSCHEDULER = FALSE;
  };

  /* Definition of application mode */
  APPMODE appmode1{};

  EVENT TouchOnEvent { MASK = AUTO; };
  EVENT TouchOffEvent { MASK = AUTO; };
  EVENT LightOnEvent { MASK = AUTO; };	
  EVENT LightOffEvent { MASK = AUTO; };

  TASK MotorcontrolTask
  {
    AUTOSTART = TRUE
    {
      APPMODE = appmode1;
    };
    PRIORITY = 2; /* Smaller value means lower priority */
    ACTIVATION = 1;
    SCHEDULE = FULL;
    STACKSIZE = 512; /* Stack size */
    EVENT = TouchOnEvent;
    EVENT = TouchOffEvent;
    EVENT = LightOnEvent;
    EVENT = LightOffEvent;
  };


  TASK EventdispatcherTask 
  {
    AUTOSTART = TRUE
    {
      APPMODE = appmode1;
    };
    PRIORITY = 1; /* Smaller value means lower priority */
    ACTIVATION = 1;
    SCHEDULE = FULL;
    STACKSIZE = 512; /* Stack size */
  };


};
\end{lstlisting}
\newpage

\section*{Part 3: Periodic Scheduling}

The third part of the lab consists of adding an ultra sonic sensor, which is used to measure distance, to the robot. The robot software is updated to use the sensor to keep a distance of around 20 cm to the object in front of it.

The updated robot software still includes the \texttt{MotorcontrolTask} from the previous part, but not the \texttt{EventdispatcherTask}, since we are now driving the robot with scheduled tasks instead of events. This also introduces changes to the \texttt{MotorcontrolTask}, which is no longer waiting for events. Three new tasks are introduced, the \texttt{ButtonpressTask}, \texttt{DisplayTask} and \texttt{UltrasonicTask}, handling the pressing of the button, displaying the light intensity and measuring the distance to the object in front of the robot, respectively.

The different tasks are run in parallell at different intervals, and they continously update the overall state of the robot to accound for the measured changes in the environment. 

The \texttt{UltrasonicTask} causes the robot to move forward if the distance to the object in front of the robot is between 20 and 100 cm. The execution interval of the UltrasonicTask is 100 ms, which means that the robot will move forward for 100 ms, after which the distance is recalculated.

The \texttt{ButtonpressTask} is run at run every 10 ms and polls the button attached to the robot. If the button is pressed, it overrides any previous move commands and moves the robot backwards.

The \texttt{DisplayTask} polls the light sensor at 100 ms intervals and prints the value to the LCD display.

\subsection*{Part 3: C-code}
\begin{lstlisting}[label=some-code,caption=part3.c,mathescape]
#include <stdlib.h>
#include <stdio.h>
#include "kernel.h"
#include "kernel_id.h"
#include "ecrobot_interface.h"

DeclareTask(DisplayTask);
DeclareTask(MotorcontrolTask);
DeclareTask(ButtonpressTask);
DeclareTask(UltrasonicTask);

DeclareCounter(SysTimerCnt);

// Kompl: Declare lock used for changing global driving commands.
DeclareResource(DriveCommandLock);

#define LIGHTSENSOR NXT_PORT_S1
#define TOUCHSENSOR NXT_PORT_S2
#define ULTRASONICSENSOR NXT_PORT_S3
#define LIGHTVAL ecrobot_get_light_sensor(LIGHTSENSOR)
#define ULTRAVAL ecrobot_get_sonar_sensor(ULTRASONICSENSOR)
#define TOUCHVAL ecrobot_get_touch_sensor(TOUCHSENSOR)
#define PRIO_IDLE 10
#define PRIO_BUTTON 20
#define SPEED 70
#define MCTP 50
#define BPTP 10
#define DTP 100
#define USTP 100

struct dc_t {
  U32 duration;
  S32 speed;
  int priority;
} dc = {0, 0, PRIO_IDLE};


void ecrobot_device_initialize() {
  ecrobot_init_sonar_sensor(ULTRASONICSENSOR) ; // initiate sensors
  ecrobot_set_light_sensor_active(LIGHTSENSOR);
}
void ecrobot_device_terminate() {
  ecrobot_term_sonar_sensor(ULTRASONICSENSOR) ; // turn sensors off
  ecrobot_set_light_sensor_inactive(LIGHTSENSOR);
}

void user_1ms_isr_type2(void){ (void)SignalCounter(SysTimerCnt); } 

void change_driving_command(int priority, int speed, int duration) {
  /* If a new driving command is sent, overwrite the previous if the new
     command has higher priority than the current. */
  if (priority >= dc.priority) {
	
	// Kompl: Only allow exclusive access the changing of global driving commands.
	GetResource(DriveCommandLock);
  
    dc.priority = priority;
    dc.speed = speed;
    dc.duration = duration;
	
	// Kompl: Release the lock when we have changed the global driving commands.
	ReleaseResource(DriveCommandLock);
  }
}

/* This function will print the light intensity on the LCD */
void display_light_intensity() {
  display_clear(1);
  display_goto_xy(0,0);
  display_string("Light:  ");
  display_unsigned(LIGHTVAL, 3);
  display_update();
}

/*  This function will take 10 light intensity values, waiting 100ms between each measurement
    and return the mean measurement value. This is used for initial calibration */
int light_calibration_proc() {
  int threshold = 0;
  for (int i=0;i<10;i++){
    threshold += LIGHTVAL;
    systick_wait_ms(100);
  }
  return threshold /= 10;
}

/* If the button is pressed, reverse for 1 second
   (actually, from pressed and for approximately one second after release time) */
TASK(ButtonpressTask){
  if (TOUCHVAL)
    change_driving_command(PRIO_BUTTON, -100, 1000); 		
  TerminateTask();
}

TASK(DisplayTask){
  display_light_intensity();
  TerminateTask();
}

/* If an obstacle is further than 21 cm move forward, if closer than 19,
   move backwards. */ 
TASK(UltrasonicTask) {
  int preferred_speed = 95;
  int d = ULTRAVAL; /* read the ultrasonic value, see #define */

  // If we are not at around 20 cm from the object in front of us.
  if (d > 21 || d < 19) {
  
    // Decrease the speed if we are close to the target distance.
    double temp = preferred_speed * ((d-20)/6);
    int speed = temp;

    // Make sure we keep within the preferred speed interval.
    if(speed < -preferred_speed)
      speed = -preferred_speed; 
    else if(speed > preferred_speed) 
      speed = preferred_speed;
    
    // Send drive command, at lower prio than reverse button.
    change_driving_command(PRIO_BUTTON - 1, speed, USTP);
    
  }

  TerminateTask();
}

TASK(MotorcontrolTask) {
  if (dc.duration > 0){ /* if we should drive... */
    nxt_motor_set_speed(NXT_PORT_A, dc.speed, 0); /* ... drive */
    nxt_motor_set_speed(NXT_PORT_B, dc.speed, 0);
    dc.duration-=MCTP; /* decrease drive time with the period of this task */
  } else if (dc.priority != PRIO_IDLE) { 
    /* if duration is negative or zero, stop */
    nxt_motor_set_speed(NXT_PORT_A, 0, 1);
    nxt_motor_set_speed(NXT_PORT_B, 0, 1);
    dc.priority = PRIO_IDLE;
  }
  TerminateTask();
}

\end{lstlisting}
\newpage
\subsection*{Part 3: Oil-file}
\begin{lstlisting}[label=some-code,caption=part3.oil,mathescape]
#include "implementation.oil"

CPU ATMEL_AT91SAM7S256
{
  OS LEJOS_OSEK
  {
    STATUS = EXTENDED;
    STARTUPHOOK = FALSE;
    ERRORHOOK = FALSE;
    SHUTDOWNHOOK = FALSE;
    PRETASKHOOK = FALSE;
    POSTTASKHOOK = FALSE;
    USEGETSERVICEID = FALSE;
    USEPARAMETERACCESS = FALSE;
    USERESSCHEDULER = FALSE;
  };

  /* Definition of application mode */
  APPMODE appmode1{};

  /* Kompl: Define a lock to be used when changing global driving commands. */
  RESOURCE DriveCommandLock {
	RESOURCEPROPERTY = STANDARD;
  };
  
  TASK MotorcontrolTask
  {
    AUTOSTART = FALSE;
    PRIORITY = 1; /* Smaller value means lower priority */
    ACTIVATION = 1;
    SCHEDULE = FULL;
    STACKSIZE = 512; /* Stack size */

  };

  TASK UltrasonicTask
  {
    AUTOSTART = FALSE;
    PRIORITY = 1; /* Smaller value means lower priority */
    ACTIVATION = 1;
    SCHEDULE = FULL;
	RESOURCE = DriveCommandLock;		/* Kompl: UltrasonicTask uses the lock. */
    STACKSIZE = 512; /* Stack size */

  };

  TASK ButtonpressTask
  {
    AUTOSTART = FALSE;
    PRIORITY = 1; /* Smaller value means lower priority */
    ACTIVATION = 1;
    SCHEDULE = FULL;
	RESOURCE = DriveCommandLock;		/* Kompl: ButtonpressTask uses the lock */
    STACKSIZE = 512; /* Stack size */
  };

  TASK DisplayTask
  {
    AUTOSTART = FALSE;
    PRIORITY = 1; /* Smaller value means lower priority */
    ACTIVATION = 1;
    SCHEDULE = FULL;
    STACKSIZE = 512; /* Stack size */
  };

	COUNTER SysTimerCnt
  {
    MINCYCLE = 1;
    MAXALLOWEDVALUE = 10000;
    TICKSPERBASE = 1; /* One tick is equal to 1msec */
  };

	ALARM cyclic_alarm
  {
    COUNTER = SysTimerCnt;
    ACTION = ACTIVATETASK
    {
        TASK = MotorcontrolTask;
    };
    AUTOSTART = TRUE
    {
        ALARMTIME = 1;
        CYCLETIME = 50;
        APPMODE = appmode1;
    };
  };

	ALARM cyclic_alarm2
  {
    COUNTER = SysTimerCnt;
    ACTION = ACTIVATETASK
    {
        TASK = ButtonpressTask;
    };
    AUTOSTART = TRUE
    {
        ALARMTIME = 1;
        CYCLETIME = 10;
        APPMODE = appmode1;
    };
  };
ALARM cyclic_alarm3
  {
    COUNTER = SysTimerCnt;
    ACTION = ACTIVATETASK
    {
        TASK = DisplayTask;
    };
    AUTOSTART = TRUE
    {
        ALARMTIME = 1;
        CYCLETIME = 100;
        APPMODE = appmode1;
    };
  };

ALARM cyclic_alarm4
  {
    COUNTER = SysTimerCnt;
    ACTION = ACTIVATETASK
    {
        TASK = UltrasonicTask;
    };
    AUTOSTART = TRUE
    {
        ALARMTIME = 1;
        CYCLETIME = 100;
        APPMODE = appmode1;
    };
  };

};

\end{lstlisting}


\end{document}

